\documentclass[12pt,a4paper]{article}
\usepackage[utf8]{inputenc}
\usepackage[slovak]{babel}
\usepackage[T1]{fontenc}
\usepackage{graphicx}
\usepackage[hidelinks]{hyperref}
\usepackage{xcolor}
\hypersetup{
	colorlinks,
	linkcolor={red!50!black},
	citecolor={blue!50!black},
	urlcolor={blue!80!black}
}
\usepackage{listings}
\usepackage{caption}
\usepackage[left=2cm,text={17cm,24cm},top=3cm]{geometry}
\usepackage{amsmath}


\newcommand{\source}[1]{\caption*{Source: {#1}} }

\bibliographystyle{czechiso}


\begin{document}
\begin{titlepage}
	\begin{center}
	    {\LARGE\textsc{Brno University of Technology}}\\
	    \smallskip
	    {\Large\textsc{Faculty of Information Technology}}\\
	    \bigskip
	    \vspace{\stretch{0.382}}
	    \smallskip
		\Huge{SFC}\\
		\huge{LSTM network demonstration}
	    \vspace{\stretch{0.618}}
	\end{center}
    {\today \hfill Svätopluk Hanzel}
\end{titlepage}

\newpage
\section{Úvod}
	Cieľom tohto projektu je implementovať aplikáciu demonštrujúcu fungovanie rekurentnej neurónovej siete s tzv. Long Short Term memory architektúrou. 

	\subsection{LSTM siete}
	Neurónové siete s LSTM architektúrou patria medzi tzv. rekurenté neurónové siete. Motiváciou pre ich použitie v kontraste s doprednými neurónovými sieťami je ich schopnosť spájať nové (aktuálne prijaté) informácie s inými dátami, ktoré boli prijaté skôr. Týmto pomáhajú bližšie emulovať napríklad ľudský mozog, nakoľko ani my - ľudia - nezačíname proces rozmýšlania odznova s každým novým slovom.
	
		\subsubsection{Architektúra}
			LSTM sieť výchadza architektúry rekurentných sietí, kde maju jednotlivé bunky spätné napojenie, čím umožňujú predávanie informácie do ďalšieho kroku (viď. obr. \ref{fig:rnn-unrolled})
			\begin{figure}[h]
				\centering
				\includegraphics[width=0.7\linewidth]{images/RNN-unrolled}
				\caption{Architektúra RNN siete s rozvinutou spätnou väzbou. \href{https://colah.github.io/posts/2015-08-Understanding-LSTMs/}{[zdroj]}}
				\label{fig:rnn-unrolled}
			\end{figure}
			Problém klasických RNN sietí spočíva v ich jednoduchosti - obsahujú len 1 tanh vrstvu, ktorá neumožnuje detailnú kontrolu informácií, ktoré sieť može potrebovať v niekoľko časových krokov ďalej.
			
			LSTM siete toto obmedzenie obchádzajú vytvorením ďalších tzv. brán (angl. gates), ktoré detailne kontrolujú tok dát.
			
			Základom každej LSTM bunky je \textit{Cell state} ($C_t$) - hodnota, ktorá ide cez všetky časové kroky a jednotlivé bunky do nej môžu "pridávať" alebo z nej môžu "odoberať" informácie práve pomocou týchto brán.
			
			\begin{figure}[h]
				\centering
				\includegraphics[width=0.7\linewidth]{images/LSTM-chain}
				\caption{Základná architektúra jednej bunky LSTM siete. \href{https://colah.github.io/posts/2015-08-Understanding-LSTMs/}{[zdroj]}}
				\label{fig:lstm-chain}
			\end{figure}
			
			
			\paragraph{Forget gate} Forget gate $f_t$ je sigmoidová vrstva, ktorá ma za úlohu rozhodnúť podľa dát z hidden vektora $h_{t-1}$ z predchádzajúcej bunky a zo vstupu $x_t$ o odstránení niektorých informácií z $C_t$.
			Výpočet potom vyzerá následovne:
			\[
			f_t = \sigma(W_f\cdot[h_{t-1},x_t] + b_i)
			\]
			Kde $W_f$ a $b_i$ sú naučené parametre siete.
			
			\paragraph{Input gate} Ďalej sa musí rozhodnúť o tom, ktoré informácie sa majú vložiť do $C_t$. Preto sa pomocou sigmoidovej vrstvy vypočíta vektor aktivácií $i_t$
			\[
			i_t = \sigma(W_i\cdot[h_{t-1},x_t] + b_i)
			\]
			Z ktorého sa ďalej spočíta vektor kandidátnych hodnôt $\tilde{C_t}$.
			\[
			\tilde{C_t} = \tanh(W_c\cdot[h_{t-1},x_t] + b_c)
			\]
			Samotná aktualizácia $C_t$ sa následnej počíta pomocou oboch týchto hodnôt a hodnoty cell state z predchádzajúceho kroku:
			\[
				C_t = f_t * C_{t-1} + i_t * \tilde{C_t}
			\]
			
			\paragraph{Output gate} Posledným krokom vo výpočte LSTM bunky je vytvoriť nový výstupný vektor $h_t$, ktorý bude následne pripojený na ďalšiu bunku v poradí. Na toto slúži brána Output gate, ktorá vypočíta z aktuálneho $C_t$ nový vektor $h_t$, ktorý je jeho filtrovanou verziou. Takto sa može zachovať nejaká informácia a predá sa ďalšej bunke.
			\[
			o_t = \sigma(W_o\cdot[h_{t-1},x_t] + b_o)
			h_t = o_t * tanh(C_t)
			\]
				
		
		\subsubsection{Backpropagation through time}
			Pre korektné natrénovanie každej neurónovej siete je dôležité vedieť aká chyba vznikla pri konkrétnom kroku učenia a napraviť jednotlivé parametre tak, aby dali v ďalšom kroku lepšie výsledky.
			
			V prípade LSTM sietí je klasický backpropagation algoritmus aktualizovaný a rozšírený pre časovú závislosť krokov. Jeho výpočet spočíva v nájdení gradientov jednotlivých parametrov. Tieto sú následne sčítané skrz všetky časové kroky a na ich základe sa vhodne aktualizujú parametre modelu. \cite{58337}
			
			Pre úplnosť uvádzam postup výpočtu gradientov počas BPTT.
			\begin{align*}
				dv_t &= \hat{y_t} - y_t \\
				dh_t &= dh'_t + W_y^T \cdot dv_t \\
				do_t &= dh_t * \text{tanh}(C_t) \\
				dC_t &= dC'_t + dh_t * o_t * (1 - \text{tanh}^2(C_t))\\
				d\bar{C}_t &= dC_t * i_t \\
				di_t &= dC_t * \bar{C}_t \\
				df_t &= dC_t * C_{t-1} \\
				\\
				df'_t &= f_t * (1 - f_t) * df_t \\
				di'_t &= i_t * (1 - i_t) * di_t \\
				d\bar{C}'_{t-1} &= (1 - \bar{C}_t^2) * d\bar{C}_t \\
				do'_t &= o_t * (1 - o_t) * do_t \\
				dz_t &= W_f^T \cdot df'_t \\
				&+ W_i^T \cdot di_t \\
				&+ W_C^T \cdot d\bar{C}_t \\
				&+ W_o^T \cdot do_t \\
				\\
				[dh'_{t-1}, dx_t] &= dz_t \\
				dC'_t &= f_t * dC_t
			\end{align*}

\newpage		
\section{Implementácia}
	V rámci tohto projektu som v jazyku C++ implementoval LSTM siet schopnú predikovať jednotlivé znaky. Projekt je rozdelený do viacerých častí - načitanie a príprava dát, práca s maticami a napokon samotná siet.
	
	\subsubsection{Načítanie a príprava dát}
	Pre zjednodušenie práce je táto zodpovednosť presunutá do hlavnej (main) funkcie. Načítanie dát spočíva hlavne vo úprave znakov na malé písmena a vytvorení slovnika, v ktorom su jednotlivé znaky mapované na čisla a takto ďalej odovzdané na čítanie.
	
	\subsubsection{Práca s maticami}
	Vzhľadom na potrebu siete pracovat s vektormi a maticami čísel som sa rozhodol implementovať vlastnú triedu pre matice - \texttt{Matrix}, ktorá implementuje rôzne operácie nad maticami; hlavne maticové násobenie, sčitanie matíc, násobenie matíc so skalármi a rôzne matematické funkcie ako napr. tangens, umocnenie čísel v matici a pod.
	
	Samotná trieda je templatovaná, čo umožnuje ukladanie rôznych dátových typov, nie len int čí double.
	
	Hlavnú limitáciu tejto triedy vidím v jej obmedzení na 2 dimenzie a neschopnosti vytvárania rôznych pohľadov na jej dáta. Toto obmedzenie by som chcel do budúcnosti odstrániť.
	
	\subsubsection{LSTM sieť}
	Samotná LSTM sieť je implementovaná v triede \texttt{LSTM}. Jej konštruktor berie ako parametre hlavne hyperparametre siete a inicializuje váhy pomocou tzv. Xavierovej inicializácie. \cite{DBLP:journals/corr/Kumar17}
	
	Ďalej má táto trieda 1 hlavnú verejnú metódu - \texttt{train}, ktorá spúšťa vo viacerých epochách túto sieť nad celým datasetom rozdeleným na viacero častí podľa dĺžky vstupu siete. V porovnaní s teoretickým model LSTM siete popísaným vyššie maju jednotlivé bunky okrem $h_t$ výstupu aj $\hat{y}_t$, ktorý je spočítaný pomocou \textit{softmax}-u a určuje pravdepodobnosti jednotlivých hodnôt zo slovníka.
	
	Po každom takomto doprednom prechode vstupu po sieti sa vypočíta chyba prechodu a spustí sa spätný (backward) prechod sieťou. Pri každom kroku spätného prechodu sa vypočítajú gradienty jednotlivých parametrov siete v tomto kroku, ktoré sa postupne sčítavajú.
	
	Na ich základe sa potom v metóde \texttt{update\_params} podľa parametrov optimalizátora a learning rate vypočítajú ich nové hodnoty.
	
	Tento proces sa opakuje pre celý dataset = jednu epochu. Po skončení epochy aplikácia vypíše na štandardný výstup loss pre epochu a ukážkový výstup,

\section{Trénovanie}
	Trénovanie som skúšal na niekoľkých datasetoch - \href{https://www.kaggle.com/kaggle/us-baby-names?select=NationalNames.csv}{menách detí}, \href{https://www.kaggle.com/kumazaki98/dinosaur-list}{dinosaurov} a na kompletnej dramatickej tvorbe Williama Shakespeara či knihách Harryho Pottera. Najlepšie výsledky som dosahoval aplikácia pri jednoduchších datasetoch so znakmi pozostávajúcich len z malých písmen anglickej abecedy, teda názvy dinosaurov.
	
	\begin{figure}[h]
		\centering
		\includegraphics[width=0.7\linewidth]{images/training-300e-lr001}
		\caption{Priebeh trénovania siete nad datasetom dinos.txt. 100 epôch a lr = 0.001}
		\label{fig:training-100e-lr0001}
	\end{figure}
	
	Pri týchto jednoslovných datasetoch sa najviac osvedčila dlžka sekvencie okolo hodnoty 20 znakov pri 20-100 epochách. Po natrénovaní dokáže sieť vygenerovať nové mená dinosaurov na základe vhodného prvého znaku. Ukážka výstupu:
	
	\texttt{
		austrosaurus
		utarapateks
		rapator
		venasugngowx
		sinornitholog
		aladromeus
		toriosaurus
		zenimaceitisaurus
		varaikenoplota
		colopteryx
		velocirapteros
		omaisaurus
		steronthodesceus
		zhenyurocoese 
	}

	Ako môžme vidieť, sieť sa naučila rozumne striedať spoluhlásky so samohláskami a používať časté prípony ako \textit{-saurus} či \textit{-us} a slová ako \textit{raptor}.

\section{Používateľský manuál}
	\subsection{Rýchle spustenie}
	Pre rýchle spustenie trénovania stačí v koreňovom adresári spustiť príkaz \texttt{make run}, ktorý zaistí preklad a spustí program s prednastavenými parametrami učenia na defaultnom datasete.
	
	\subsection{Preklad}
	Aplikácia používa štandard C++11 a nemá žiadne externá závislosti, takže by mala byť preložiteľná na všetkých strojoch s nainštalovaným prekladačom gcc verzie 4.8 a vyššie.
		
	Na samotný preklad má následne aplikácia priloženy \texttt{Makefile} súbor so všetkými potrebnými targetmi pre preklad. Na samostatné preloženie stačí spustiť príkaz \texttt{make}. Výsledná binárka potom bude uložená v adresári \texttt{dist}.

	\subsection{Používanie}	
	Aplikácia má niekoľko prepínačov, ktoré umožňujú zmeniť hyperparametre siete alebo parametre učenia. Kompletný zoznam aplikácia vypíše po spustení s prepínačom \texttt{-h}. Najdoležitejší je prepínač \texttt{-f}, ktorý udáva cestu k datasetu.
	
	Po spustení sa automaticky spustí učenie siete podľa parametrov, po každej epoche vypíše aktuálnu hodnotu loss a vypíše ukážku textu. Po skončení učenia program končí.
	

\section{Záver}
Implementovaná verzia je funkčnou LSTM sieťou s učením a BP. Možné vylepšenia vidím hlavne v oblasti práce s multidimenzionálnymi dátami. Mnou vytvorená trieda \texttt{Matrix} je príliš neefektívna, čo má za následok dlhé a menej spoľahlivé učenie.

\bibliography{references}
\end{document}
